\chapter{Podsumowanie}
\label{cha:Podsumowanie i kierunki dalszych prac}

W~pracy przedstawiono sprzętowo-programową realizację systemu wspomagającego autonomiczne lądowanie drona.
System zrealizowano w~układzie Zynq SoC na karcie ZYBO Z7-20. 
Część rekonfigurowalna pozwoliła na~szybkie przetwarzanie obrazu, natomiast system procesorowy umożliwił sprawną realizację algorytmu sterowania i~wysyłanie komend do~sterownika drona. 

Zaimplementowano dwie wersje systemu wizyjnego: z~indeksacją jednoprzebiegową i identyfikacją znacznika na podstawie koloru i~współczynnika kształtu oraz z~wyznaczaniem środka ciężkości i~identyfikacją markera przy użyciu jego barwy. 
Poprawną segmentację na~podstawie koloru przy niewielkiej zależności od~oświetlenia umożliwiła binaryzacja w~przestrzeni barw YCbCr, w~oparciu o~składowe Cb i~Cr. 
Zaprojektowano również taki znacznik, aby ułatwić jego wykrycie. 
Zaimplementowanie modułów mediany i~otwarcia pozwoliło na~filtrację obrazu.

Wprowadzenie do~układu sygnału z~czujnika lidar umożliwiło dokładną kontrolę wysokości.
%TODO radip
Zaimplementowano także regulację PID, której celem jest ustawienie drona nad znacznikiem. 
Eksperymenty pokazały możliwość wykonania symulowanego lotu testowego składającego się ze~startu, regulacji położenia i~lądowania w~wyznaczonym miejscu.
Przeprowadzone testy dowiodły również możliwości wysyłania do~sterowanika komend, między innymi dotyczących zmiany kierunku i~szybkości lotu. 

Kolejnym celem powinno być wykonanie lotu testowego (start -- regulacja położenia -- lądowanie) przy sterowaniu na~podstawie wizyjnego sprzężenia zwrotnego. 
Etapem pośrednim powinno być przeprowadzenie testu pozwalającego ustalić, jaką najmniejszą prędkość akceptuje autopilot, jaka największa prędkość powoduje ruch drona bez wyraźnego przechylenia i~pochylenia oraz z~jaką częstotliwością mogą być wysyłane komendy. 
Te~informacje pozwoliłyby na~dobór nastaw regulatora. 
W~przypadku niemożliwości realizacji komend z~odpowiednią częstotliwością rozważyć można inną koncepcję: na~podstawie wysokości i~aktualnej pozycji znacznika na~obrazie można byłoby wyznaczać od~razu przesunięcie (zamiast prędkości), które doprowadziłoby dron ponad znacznik. Zadawanie przesunięcia również jest wspierane przez oprogramowanie ArduPilot. Takie podejście wymagałoby przeprowadzenia testów odległości od~znacznika w~terenie, w~zależności od~tej odległości na~obrazie. %TODO2 Nie bardzo to rozumiem... (doprecyzowane)

Powyższe działania możliwe są~przy zastosowaniu komponentów użytych w~projekcie. 
Użycie kamery o~większym kącie widzenia pozwoliłoby jednak na~detekcję lądowiska z~większej odległości.
Innym kierunkiem dalszych pracach może być próba wykonania lądowania na~poruszającej się platformie. 
Analiza literatury naukowej dostarczyła informacji co~do~niezbędnych działań, jakie musi wykonywać taki system. 
Są~to: śledzenie lądowiska, predykcja ruchu lądowiska, generacja trajektorii pozwalających dotrzeć do~celu, wybór jednej z~dopuszczalnych oraz realizacja odpowiedniego sterowania.

%TODO + Dodatek A. Spis zawartości CD, która będzie dla mnie (nie dziekanatu).
\appendix
\chapter{Spis zawartości płyty CD}
Na dołączonej do~pracy płycie CD znajdują się następujące foldery:
\begin{itemize}
	\item Matlab -- zawiera pliki modeli programowych, testów, obróbki obrazów oraz zdjęcia niezbędne do ich uruchomienia,
	\item Vivado -- zawiera projekt główny oraz testowy, wykorzystywany do~komunikacji z~autopilotem,
	\item Tex -- zawiera pliki tekstowe \LaTeX wraz z~rysunkami pojawiającymi się w~tekście.
\end{itemize}
%---------------------------------------------------------------------------
