\chapter{Metody przetwarzania obrazu i sposoby sterowania w procesie autonomicznego lądowania drona}
%Przegląd systemów wizyjnych wspomagających autonomiczne lądowanie drona
\label{cha:Metody przetwarzania obrazu i sposoby sterowania w procesie autonomicznego lądowania drona}

%TODO Tutaj jakieś "zagajenie" tzn. opisać proces takiego lądowania i wyróżnić te dwa podejścia (stałe, ruchome). Nawet jakiś rysunek poglądowu 

W przypadku lądowania na oznaczonym stacjonarnym lądowisku, głównym problemem jest skuteczna detekcja znacznika. 
Do realizacji tego zadania niezbędne jest wykonanie szeregu kroków.
Przykładowe rozwiązanie z~pracy \cite{Rings} obejmuje:
\begin{itemize}
	\item zamianę obrazu kolorowego na obraz w skali szarości,
	\item binaryzację ze stałym progiem,
	\item indeksację,
	\item odrzucenie obiektów o~liczbie pikseli mniejszej niż zadany próg,
	\item binaryzację ze stałym progiem, %TODO rola drugiej binaryzacji niejsane
	\item identyfikację znacznika.
\end{itemize}

Wartości progów binaryzacji oraz odrzucenia małych obiektów zostały dobrane eksperymentalnie.\\ 
Aby dodatkowo wyeliminować biały szum oraz zakłócenia typu ,,sól i~pieprz'' w~pracy \cite{H_median} wprowadzono operację mediany z~oknem 3x3 na obrazie w skali szarości. 
Dodatkowo wprowadzono ograniczenie na maksymalny rozmiar obiektu. 
W algorytmie przedstawionym w pracy \cite{FPGA} binaryzacji dokonano w~przestrzeni barw HSV (ang. \textit{Hue Saturation Value}), na~podstawie składowych S i~V. 
Pozwoliło to na przeprowadzenie segmentacji niezależnej od koloru. 
Przed rozpoznawaniem kształtów, na~obrazie binarnym dokonano kolejno erozji, detekcji krawędzi oraz dylatacji. 
Wykonanie erozji pozwoliło na eliminację z~obrazu małych grup pikseli.\par %TODO a po co krwędzie ?

Istotnym elementem planowania autonomicznego lądowania drona jest wybór znacznika lądowiska. 
W~pracy \cite{Falanga} przedstawiono marker złożony z trzech figur: kwadratu, koła i~krzyża. %TODO a może jakieś zdjęcie ? z tego artykułu, z innych też - porównanie użytych znaczników ?
Zaimplementowany został algorytm detekcji lądowiska polegający na kolejnym wykrywaniu wymienionych figur, co pozwoliło na coraz lepsze określenie położenia celu.\\
Prostsze rozwiązanie zostało pokazane w~pracy\cite{H}, gdzie znacznik ma kształt litery H. 
Odległość drona od lądowiska obliczana jest na podstawie liczby pikseli dzielących środek ciężkości znacznika od~środka obrazu. 
Opisano tam również metodę wyznaczania orientacji drona względem takiego markera. 
Polega ona na~znalezieniu piksela najbardziej odległego od środka ciężkości.\\ %TODO jak dla mnie niejasne

Inny znacznik został użyty w pracy \cite{Rings}. 
Składa się on z~czterech pierścieni na czarnym tle. 
Każdy pierścień posiada unikalny stosunek promienia wewnętrznego do zewnętrznego i~stanowi oddzielny obiekt dla systemu wizyjnego. 
Obiekty, które nie mają dokładnie jednej dziury wewnątrz, są~pomijane. 
Każdy z~pozostałych jest ostatecznie identyfikowany za pomocą współczynnika kształtu. 
Zaletą takiego znacznika jest możliwość dołożenia kolejnych, większych pierścieni, jeśli lądowisko ma być widoczne z~większej wysokości.\par %TODO fotos
 
Implementacja toru wizyjnego w systemie potokowym na~platformie sprzętowej rodzi dodatkowe trudności. Bez dołączenia dodatkowej pamięci RAM niemożliwe są działania na~całej ramce obrazu. Jest to uciążliwe przy wykonywaniu operacji kontekstowych, wymagających znajomości otoczenia piksela. Problem rozwiązano w \cite{FPGA}, gdzie przedstawiono koncepcję realizacji takich operacji przy użyciu tablicy 3x3 i~bufora.\par %TODO czego dotyczyła dokładnie ta pracy, bo wydaje się zbędna (na tym etapie)

W procesie autonomicznego lądowania drona, na~podstawie wyznaczonej odległości od celu, do UAV wysyłane są sygnały sterujące. 
W~pracy \cite{Sudevan} wykorzystano trzy regulatory PID, do kontroli prędkość drona. %TODO dlaczego trzy 
Równocześnie minimalizowana jest każda ze~składowych wektora położenia. 
W~pracy \cite{Rings} zwrócono uwagę na błędy spowodowane pochyleniem i~przechyleniem drona. 
Przed wysłaniem sygnału do regulatora wykonywana jest korekta uchybu na podstawie pomiaru kąta pochylenia i~przechylenia.\par %TODO jakieś szczegóły 

Metody lądowania na statycznym lądowisku nie zawsze znajdują zastosowanie do śledzenia poruszającej się platformy. %TODO zdanie niejasne - do zmiany 
W~przypadku, gdy lądowisko przestaje być widoczne, możliwe jest przewidywanie jego ruchu.  
W~\cite{Falanga} użyto algorytmu pozwalającego na predykcję zachowania platformy. 
Wykorzystano model dynamiczny celu oraz rozszerzony filtr Kalmana. 
Spośród możliwych trajektorii dotarcia do celu wybierana jest najbardziej efektywna pod względem energetycznym.\par

%TODO A coś więcej o tych ruchomych ?

 
Podsumowując, wykonanie lądowania na statycznym lądowisku wymaga skutecznej detekcji znacznika znajdującego, który określa jego lokalizację. 
Znalezienie względnej pozycji markera umożliwia przekazanie informacji o uchybie do regulatora sterującego dronem. 
Lądowanie na poruszającej się platformie może zostać wykonane w~podobny sposób, jednak wprowadzenie przewidywania ruchu lądowiska daje odporność na zaniki widoczności. %TODO nie wiem czy "zanik" to podstawowy problem.
Wymaga to jednak użycia znacznie bardziej zaawansowanych narzędzi. %TODO trochę zbyt ogólnie.




%---------------------------------------------------------------------------




