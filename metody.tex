\chapter{Metody przetwarzania obrazu i sposoby sterowania w procesie autonomicznego lądowania drona}
%Przegląd systemów wizyjnych wspomagających autonomiczne lądowanie drona
\label{cha:Metody przetwarzania obrazu i sposoby sterowania w procesie autonomicznego lądowania drona}
W przypadku lądowania na stacjonarnym celu głównym problemem jest skuteczna detekcja znacznika znajdującego się na~lądowisku. Do realizacji tego zadania niezbędne jest wykonanie szeregu etapów przetwarzania wizyjnego. W~pracy \cite{Rings} przedstawione zostały następujące kroki przetwarzania obrazów:
\begin{itemize}
	\item zamiana obrazu kolorowego na obraz w skali szarości,
	\item binaryzacja ze stałym progiem,
	\item indeksacja,
	\item odrzucenie obiektów o~liczbie pikseli mniejszej niż zadany próg,
	\item binaryzacja ze stałym progiem,
	\item identyfikacja znacznika.
\end{itemize}

Progi binaryzacji oraz odrzucenia małych obiektów zostały dobrane eksperymentalnie.\\ 
Aby dodatkowo wyeliminować biały szum oraz zakłócenia typu ,,sól i pieprz'' w \cite{H_median} wprowadzono operację mediany z~oknem 3x3 na obrazie w skali szarości. Aby~usunąć zbyt duże obiekty, zdecydowano się również wprowadzić górną granicę liczby pikseli należących do obiektu. W algorytmie przedstawionym w pracy \cite{FPGA} binaryzacji dokonano w~przestrzeni HSV, na~podstawie składowych S i~V. Pozwoliło to na przeprowadzenie segmentacji niezależnej od koloru. Przed rozpoznawaniem kształtów, na~obrazie binarnym dokonano kolejno erozji, detekcji krawędzi oraz dylatacji. Wykonanie erozji pozwoliło na eliminację z~obrazu małych grup pikseli.\par 
Istotnym elementem planowania autonomicznego lądowania drona jest wybór charakterystyki znacznika. W~\cite{Falanga}  przedstawiono marker złożony z trzech figur: kwadratu, koła i~krzyża. Zaimplementowany został algorytm detekcji lądowiska polegający na kolejnym wykrywaniu figur i~uzyskiwaniu coraz lepszej znajomości pozycji celu.\\
Prostsze rozwiązanie zostało pokazane w \cite{H}, gdzie znacznik ma kształt litery H. Odległość drona od lądowiska obliczana jest na podstawie liczby pikseli dzielących środek ciężkości znacznika od~środka obrazu. Opisano tam również metodę wyznaczania orientacji drona względem takiego markera. Polega ona na~znalezieniu piksela najbardziej odległego od środka ciężkości.\\
Inny znacznik został użyty w \cite{Rings}. Składa się on z~czterech pierścieni na czarnym tle. Każdy pierścień posiada unikalny stosunek promienia wewnętrznego do zewnętrznego i~stanowi oddzielny obiekt dla systemu wizyjnego. Obiekty, które nie mają dokładnie jednej dziury wewnątrz, są~pomijane. Każdy z~pozostałych jest ostatecznie identyfikowany za pomocą współczynnika kształtu. Zaletą takiego znacznika jest możliwość dołożenia kolejnych, większych pierścieni, jeśli lądowisko ma być widoczne z~większej wysokości.\par 
Implementacja toru wizyjnego w systemie potokowym na~platformie sprzętowej rodzi dodatkowe trudności. Bez dołączenia dodatkowej pamięci RAM niemożliwe są działania na~całej ramce obrazu. Jest to uciążliwe przy wykonywaniu operacji kontekstowych, wymagających znajomości otoczenia piksela. Problem rozwiązano w \cite{FPGA}, gdzie przedstawiono koncepcję realizacji takich operacji przy użyciu tablicy 3x3 i~bufora.\par 
W procesie autonomicznego lądowania drona, na~podstawie wyznaczonej odległości od celu, do UAV wysyłane są sygnały sterujące. W \cite{Sudevan} wykorzystano trzy regulatory PID, zadające prędkość drona. Równocześnie minimalizowana jest każda ze~składowych wektora położenia. \cite{Rings} zwraca uwagę na błędy spowodowane pochyleniem i~przechyleniem drona. Przed wysłaniem sygnału do regulatora wykonywana jest korekta uchybu na podstawie pomiaru kąta pochylenia i~przechylenia.\par 
Metody lądowania na statycznym lądowisku nie zawsze znajdują zastosowanie do śledzenia poruszającej się platformy. W~przypadku, gdy lądowisko przestaje być widoczne, możliwe jest przewidywanie jego ruchu.  W~\cite{Falanga} użyto algrytmu pozwalającego na predykcję zachowania platformy. Wykorzystano model dynamiczny celu oraz rozszerzony filtr Kalmana. Spośród możliwych trajektorii dotarcia do celu wybierana jest najlepsza pod względem energetycznym.\par 
Podsumowując, wykonanie lądowania na statycznym lądowisku wymaga skutecznej detekcji znacznika znajdującego się na platformie. Znalezienie względnej pozycji markera umożliwia przekazanie informacji o uchybie do regulatora sterującego dronem. Lądowanie na poruszającej się platformie może zostać wykonane w~podobny sposób, jednak wprowadzenie przewidywania ruchu lądowiska daje odporność na zaniki widoczności. Wymaga to jednak użycia znacznie bardziej zaawansowanych narzędzi.




%---------------------------------------------------------------------------




